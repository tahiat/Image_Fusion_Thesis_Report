\section{Some new section}

You can summarize in steps:

\begin{description}\itemsep1pt \parskip0pt \parsep0pt
  \item[Step 1] step 1.
  \item[Step 2] some steps with sub-steps
  \begin{description}
    \item[Step 2.1] sub-step 1
    \item[Step 2.2] sub-step 2
    \begin{description}
      \item[Step 2.2.1] sub-sub-step 1
      \item[Step 2.2.2] sub-sub-step 2
    \end{description}
  \end{description}
  \item[Step 3] Step something.
\end{description}

\section{Flowchart}

% be careful using flowcharts, use proper help from internet
Following Figure \ref{fig:Flowchart}.

\begin{figure}[!h]
\centering
\scriptsize
%% insert newline inside a node by " \\ " , not "\\"
\begin{tikzpicture}
  \node[startstop] (start) {Start};
  \node[io] (cache) [below=0.6cm of start, text width=5cm] {A};
  \node[process] (QTD) [below=0.6cm of cache, text width=3.5cm] {B};
  \node[process] (updtTmp) [below=0.6cm of QTD, text width=3.5cm] {C};
  \node[process] (feasibility) [below=0.6cm of updtTmp, text width=3.5cm] {D};
  \node[process] (fCheck) [below=0.6cm of feasibility] {E};
  \node[decision] (isFsbl) [below=0.6cm of fCheck, text width=1.5cm] {F};
  \node[process] (rmvInfsbl) [left=0.8cm of isFsbl, text width=1.5cm] {I};
  %% for loopback to fCheck
  \node [coordinate, left=2cm of fCheck] (point) {};  %% Coordinate on bottom and middle
  \node[process] (update) [below=0.6cm of isFsbl, text width=2.5cm] {J};
  \node[decision] (allDone) [below=0.6cm of update, text width=2cm] {K};
  %% for 2nd column
  \node [coordinate, left=3cm of allDone] (bottomL) {};  %% Coordinate on bottom and middle
  \node [coordinate, left=2.9cm of fCheck] (upL) {};  %% Coordinate on right and middle
  %% for 2nd column
  \node [coordinate, below=1cm of allDone] (left) {};  %% Coordinate on bottom and middle
  \node [coordinate, right=3.5cm of left] (bottom) {};  %% Coordinate on right
  \node [coordinate, above=10.5cm of bottom] (up) {};  %% Coordinate on right and middle
  \node [coordinate, right=3cm of up] (right) {};
  \node[process] (nxtItem) [above right=6.7cm and 4cm of allDone, text width=3cm] {L};
  \node[process] (encode) [below=0.6cm of nxtItem, text width=3cm] {M};
  \node[process] (broadcast) [below=0.6cm of encode, text width=3.5cm] {N};
  \node[decision] (allItemDone) [below=0.6cm of broadcast, text width=2cm] {O};
  %% for loopback to nxtItem
  \node [coordinate, right=1cm of nxtItem] (pointR) {};  %% Coordinate on bottom and middle
  \node[process] (updtTemp) [below=0.6cm of allItemDone, text width=3.5cm] {P};
  \node[process] (updtCache) [below=0.6cm of updtTemp, text width=3.5cm] {Q};
  \node[decision] (leaveRSU) [below=0.6cm of updtCache, text width=2cm] {R};
  %% for loopback to QTD
  \node [coordinate, right=8cm of QTD] (pointRR) {};  %% Coordinate on bottom and middle
  \node[process] (trnsCache) [below=0.6cm of leaveRSU, text width=3.5cm] {S};

  \draw [arrow] (start) -- (cache);
  \draw [arrow] (cache) -- (QTD);
  \draw [arrow] (QTD) -- (updtTmp);
  \draw [arrow] (updtTmp) -- (feasibility);
  \draw [arrow] (feasibility) -- (fCheck);
  \draw [arrow] (fCheck) -- (isFsbl);
  \draw [arrow] (isFsbl) -- node[anchor=south] {no} (rmvInfsbl);
  %\draw [arrow] (rmvInfsbl) |- (fCheck);
  \draw [arrow] (isFsbl) -- node[anchor=east] {yes} (update);
  \draw [arrow] (update) -- (allDone);
  \draw [arrow] (allDone) -- node[anchor=south] {no} (bottomL) -- (upL) edge (fCheck);
  \draw [arrow] (allDone.south) -- node[anchor=west] {yes} (left) -- (bottom) -- (up) -- (right) --  (nxtItem);
  \draw [arrow] (nxtItem) -- (encode);
  \draw [arrow] (encode) -- (broadcast);
  \draw [arrow] (broadcast) -- (allItemDone);
  \draw [arrow] (allItemDone.east) -| node[anchor=south east] {no}(pointR) -- (nxtItem);
  \draw [arrow] (allItemDone.south) -- node[anchor=west] {yes} (updtTemp);
  \draw [arrow] (updtTemp) -- (updtCache);
  \draw [arrow] (updtCache) -- (leaveRSU);
  \draw [arrow] (leaveRSU.east) -| node[anchor=south east] {no} (pointRR) -- (QTD);
  \draw [arrow] (leaveRSU.south) -- node[anchor=west] {yes} (trnsCache);

  %\path[draw,->] (start) edge (cache)
                 %(cache) edge (QTD)
                 %(QTD) edge (updtTmp)
                 %(updtTmp) edge (feasibility)
                 %(feasibility) edge (fCheck)
                 %(fCheck) edge (isFsbl)
                 %(isFsbl.west) edge node[anchor=south] {no} (rmvInfsbl)
                 %(rmvInfsbl) |- (fCheck)
                 %(isFsbl.south) -- node[anchor=east] {yes} (update)
                 %(update) -- (allDone)
                 %(allDone) -- node[anchor=south] {no} (bottomL) -- (upL) edge (fCheck)
                 %(allDone.south) -- node[anchor=west] {yes} (left) -- (bottom) -- (up) -- (right) --  (nxtItem)
                 %;

\end{tikzpicture}
\linebreak
\linebreak
\caption{Flowchart of something} \label{fig:Flowchart}
\end{figure}


\section{Table Example}
Table \ref{table:algoExample}.

% Here's a table example

\begin{table}[ht]
\caption{Particulars of arrived queries at $RSU_i$.}  % title of Table
\scriptsize
\centering          % used for centering table
\begin{tabular}{p{3cm} p{4cm} p{4cm} p{3cm}}    % centered columns (4 columns)
\hline\hline %inserts double horizontal lines
Query & Required data items & Data items in cache & Remaining QTD \\[0.5ex]  % inserts table
%heading
\hline    % inserts single horizontal line
$Q_1$ & $d_1,d_2,d_5$ & $d_3,d_4,d_6$ & 3 \\      % inserting body of the table
$Q_2$ & $d_2,d_3, d_6$ & $d_1,d_4,d_5$ & 3 \\
$Q_3$ & $d_1,d_3$ & $d_2,d_4,d_5, d_6$ & 2 \\
$Q_4$ & $d_2,d_6$ & $d_1,d_3,d_4,d_5$ & 2 \\
$Q_5$ & $d_4,d_5$ & $d_1,d_2,d_3,d_6$ & 2 \\[1ex]  % [1ex] adds vertical space
\hline          %inserts single line
\end{tabular}
\label{table:algoExample}    % is used to refer this table in the text
\end{table}